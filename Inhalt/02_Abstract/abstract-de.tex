\renewcommand{\abstractname}{Abstract} % Veränderter Name für das Abstract
\begin{abstract}
\begin{addmargin}[1.5cm]{1.5cm}        % Erhöhte Ränder, für Abstract Look
\thispagestyle{plain}                  % Seitenzahl auf der Abstract Seite

\begin{center}
\small\textit{- Deutsch -}             % Angabe der Sprache für das Abstract
\end{center}

\vspace{0.25cm}

Roboter unterstützen vermehrt bei der Arbeit in Lagerhallen oder übernehmen diese gar vollständig.
Doch diese Roboter müssen entwickelt und die Systeme Kunden präsentiert werden, bevor sich diese zum Kauf und Einrichtung in ihrer persönlichen Umgebung entschließen.
Zu diesem Zweck ist die Möglichkeit unerlässlich wertvoll, eine komplette Umgebung performant simulieren zu können.

\vspace{0.25cm}

Diese Arbeit beschäftigt sich mit der Erstellung eines Services, welcher die \acs{OData}-Schnittstelle eines SAP \acl{EWM} Systems detailgetreu nachbilden soll.
Ein solcher Service existiert bereits, jedoch ist die aktuelle Fassung aufwändig gestaltet und aufgrund ihrer komplexen Architektur mit Performance Overhead verbunden, weshalb die Software von Grund auf neu konzeptioniert werden soll.
Dieser Service soll später unter anderem dazu dienen, in Unity simulierte Roboter mit Pseudoaufträgen zu versorgen.


\end{addmargin}
\end{abstract}