\chapter{Mathematik}

\section{Text}
Hier steht ein beispielhafter Text bei dem nun auf eine sehr bekannte und durchaus vertraute Formel im Text direkt mit $c = \sqrt{a^2 + b^2}$ eingegangen wird. Dabei können auch Winkel wie: $\alpha, \beta, \gamma$ gerne verwendet werden. Weiterhin werden hier nur Formeln im Bereich der $\mathbb{N}$ dargestellt, gerne können diese aber durch Formeln aus diesem Bereich der $\mathbb{R}$ ergänzt werden.

\section{Formeln}
Beachte bei Formeln keine konkreten Werte anzugeben, sondern die Formel stets nur wie in der Literatur nur als Größengleichungen anzugeben.
\begin{align}
	\sum_{n=0}^{\infty}x=b+n\\
	\frac{b*x}{c} = y
\end{align}

Trotz unterschiedlicher Länge kann man die Gleichheitszeichen auf der gleichen Höhe anbringen wie in \autoref{eq:Gleichung1} und \autoref{eq:Gleichung2} dargestellt, zusätzlich kann man diese auch mit Informationen versehen wie in Formel \autoref{eq:Gleichung3} zu sehen.

\begin{align}
	\label{eq:Gleichung1} a + b &= c\\
	\label{eq:Gleichung2} 5c + 3f &= 4h\\ 
	\label{eq:Gleichung3} \overbrace{5y}^{y = 0} + \underbrace{42x}_{x = 1} &= b
\end{align}

\section{Arrays und Matrizen}

\begin{center}
	\(
	\begin{array}{lc|r}
		a&b&c\\
		\hline
		x&y&z\\
		c&a&b
	\end{array}
	\)
\end{center}

\begin{equation}
	\begin{array}{lcl}
		z & = & a \\
		a + b & = & c \\
		f(x,y,z) & = & x + y + z
	\end{array}
\end{equation}

Hier noch ein paar Matrizen Beispiele in \LaTeX{}.

\begin{align}
	\begin{pmatrix}
		a_{11}	& \dots   & a_{1n}\\
		\vdots	& \ddots  & \vdots\\
		a_{n1}	& \dots   & a_{nn}\\
	\end{pmatrix}
	\\[0.4cm]
	\begin{bmatrix} 
		100&250\\
		300&499
	\end{bmatrix}
	\\[0.4cm]
	\begin{Bmatrix} 
		100&250\\
		300&499
	\end{Bmatrix}
	\\[0.4cm]
	\begin{Vmatrix}
		100&250\\
		300&499
	\end{Vmatrix}
\end{align}

\section{Klammern und Kästchen}

\begin{center}
	\( \Vert x\Vert_{p}=
	\left(
	\sum_{i=1}^{n} | x_{i} |^{p}
	\right)^{\frac{1}{p}} \)
\end{center}

\begin{center}
	\( \left.
	\begin{array}{lc|r}
		a&b&c\\
		\hline
		x&y&z\\
		c&a&b
	\end{array}
	\right\}
	\Rightarrow z,b \)
\end{center}

\begin{equation*}
\mbox{
	\boxed{
		\sin^2\varphi+\cos^2\varphi=1
	}
}
\end{equation*}

