\chapter*{Formelverzeichnis}
\addcontentsline{toc}{chapter}{Formelverzeichnis} % Hinzufügen zum Inhaltsverzeichnis 

% Definition des neuen Befehls für das Einfügen der Abkürzung der Einheit
\newcommand{\acrounit}[1]{
  \acroextra{\makebox[18mm][l]{\si[per-mode=fraction,fraction-function=\sfrac]{#1}}}
}
\begin{acronym}[dmin] % längstes Kürzel wird verw. für den Abstand zw. Kürzel u. Text

	% Alphabetisch selbst sortieren - nicht verwendete Formeln rausnehmen!
	% Allgemein: \acro{KÜRZEL}[ABKÜRZUNG]{\acrounit{SI-EINHEIT}BESCHREIBUNG}

	\acro{A}[\ensuremath{A}]{\acrounit{mm^2}Fläche}	
	\acro{D}[\ensuremath{D}]{\acrounit{mm}Werkstückdurchmesser}	
	\acro{dmin}[\ensuremath{d\textsubscript{min}}]{\acrounit{mm}kleinster Schaftdurchmesser}	
	\acro{L1}[\ensuremath{L\textsubscript{1}}]{\acrounit{mm}Länge des Werkstückes Nr. 1}	
	\acro{Fwinkel}[]{\acrounit{Grad}Freiwinkel}	
	\acro{Kwinkel}[]{\acrounit{Grad}Keilwinkel}

\end{acronym}
